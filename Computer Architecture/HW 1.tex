\documentclass[]{article}
\usepackage{lmodern}
\usepackage{amssymb,amsmath}
\usepackage{ifxetex,ifluatex}
\usepackage{fixltx2e} % provides \textsubscript
\ifnum 0\ifxetex 1\fi\ifluatex 1\fi=0 % if pdftex
  \usepackage[T1]{fontenc}
  \usepackage[utf8]{inputenc}
\else % if luatex or xelatex
  \ifxetex
    \usepackage{mathspec}
  \else
    \usepackage{fontspec}
  \fi
  \defaultfontfeatures{Ligatures=TeX,Scale=MatchLowercase}
\fi
% use upquote if available, for straight quotes in verbatim environments
\IfFileExists{upquote.sty}{\usepackage{upquote}}{}
% use microtype if available
\IfFileExists{microtype.sty}{%
\usepackage[]{microtype}
\UseMicrotypeSet[protrusion]{basicmath} % disable protrusion for tt fonts
}{}
\PassOptionsToPackage{hyphens}{url} % url is loaded by hyperref
\usepackage[unicode=true]{hyperref}
\hypersetup{
            pdftitle={"Computer Architecture - HW 1"},
            pdfauthor={"Connor Finley"},
            pdfborder={0 0 0},
            breaklinks=true}
\urlstyle{same}  % don't use monospace font for urls
\usepackage{longtable,booktabs}
% Fix footnotes in tables (requires footnote package)
\IfFileExists{footnote.sty}{\usepackage{footnote}\makesavenoteenv{long table}}{}
\usepackage{graphicx,grffile}
\makeatletter
\def\maxwidth{\ifdim\Gin@nat@width>\linewidth\linewidth\else\Gin@nat@width\fi}
\def\maxheight{\ifdim\Gin@nat@height>\textheight\textheight\else\Gin@nat@height\fi}
\makeatother
% Scale images if necessary, so that they will not overflow the page
% margins by default, and it is still possible to overwrite the defaults
% using explicit options in \includegraphics[width, height, ...]{}
\setkeys{Gin}{width=\maxwidth,height=\maxheight,keepaspectratio}
\IfFileExists{parskip.sty}{%
\usepackage{parskip}
}{% else
\setlength{\parindent}{0pt}
\setlength{\parskip}{6pt plus 2pt minus 1pt}
}
\setlength{\emergencystretch}{3em}  % prevent overfull lines
\providecommand{\tightlist}{%
  \setlength{\itemsep}{0pt}\setlength{\parskip}{0pt}}
\setcounter{secnumdepth}{0}
% Redefines (sub)paragraphs to behave more like sections
\ifx\paragraph\undefined\else
\let\oldparagraph\paragraph
\renewcommand{\paragraph}[1]{\oldparagraph{#1}\mbox{}}
\fi
\ifx\subparagraph\undefined\else
\let\oldsubparagraph\subparagraph
\renewcommand{\subparagraph}[1]{\oldsubparagraph{#1}\mbox{}}
\fi

% set default figure placement to htbp
\makeatletter
\def\fps@figure{htbp}
\makeatother


\title{"Computer Architecture - HW 1"}
\author{"Connor Finley"}
\date{"September 2, 2017"}

\begin{document}
\maketitle

\subsection{1}\label{header-n4699}

\begin{quote}
Consider three different processors P1, P2 and P3 executing the same
instruction set.

\begin{itemize}
\item
  P1 has a 3 GHz clock rate and a CPI of 1.5.
\item
  P2 has a 2.5 GHz clock rate and a CPI of 1.0.
\item
  P3 has a 4.0 GHz clock rate and has a CPI of 2.2.
\end{itemize}
\end{quote}

\subsubsection{a.}\label{header-n4713}

\begin{quote}
Which processor has the highest performance expressed in
\textbf{instructions per second}?
\end{quote}

\begin{quote}
1 GHz = 10\^{}9 cycles/sec

3 GHz = 3 * 10\^{}9 cycles/sec

1.5 CPU cycles / instruction count
\end{quote}

\(\frac{3 \times 10^9 \;\text{cycles}}{1 \;\text{second}}, \quad \frac{1.5 \; \text{cycles}}{1  \; \text{instruction}}, \quad \frac{\text{? instructions}}{1 \; \text{second}}\)

\(\text{P1}: \quad \frac{3 \times 10^9 \;\text{cycles}}{1 \;\text{second}} \div \frac{1  \; \text{instruction}}{1.5 \; \text{cycles}}\)
\(= \frac{3 \times 10^9 \; \text{instructions}}{1.5 \; \text{seconds}} = 2 \times 10^9 \; \text{instructions per second}\)

\begin{center}\rule{0.5\linewidth}{\linethickness}\end{center}

\(\text{P2}: \quad \frac{2.5 \times 10^9 \;\text{cycles}}{1 \;\text{second}} \div \frac{1  \; \text{instruction}}{1 \; \text{cycles}}\)
\(=2.5 \times 10^9 \; \text{instructions per second}\)

\begin{center}\rule{0.5\linewidth}{\linethickness}\end{center}

\(\text{P3}: \quad \frac{4 \times 10^9 \;\text{cycles}}{1 \;\text{second}} \div \frac{1  \; \text{instruction}}{2.2 \; \text{cycles}}\)
\(=1.82 \times 10^9 \; \text{instructions per second}\)

\begin{center}\rule{0.5\linewidth}{\linethickness}\end{center}

\(P3 = 1.82 \times 10^9 < P1 = 2 \times 10^9 < P2 = 2.5 \times 10^9\)

P2 has the highest performance with \(2.5  \times 10^9\) instructions
per second.

\subsubsection{b.}\label{header-n4742}

\begin{quote}
If the processors each execute a program in 10 seconds, find the number
of cycles and the number of instructions.
\end{quote}

\(P1 = 10 \; \text{seconds} \times 3 \; \text{GHz} = 10 \; \text{seconds} \times (3 \times 10^9 \; \text{cycles per second}) = 3 \times 10^{10} \; \text{cycles}\)

\(\frac{3 \times 10^{10} \; \text{instructions}}{1.5 \; \text{seconds}} = 2 \times 10^{10} \; \text{instructions per second}\)

\(10 \;\text{seconds} \times (2 \times 10^{10} \;\text{instructions per second}) = 2 \times 10^{11} \;\text{instructions}\)

\begin{center}\rule{0.5\linewidth}{\linethickness}\end{center}

\(P2 = 10 \; \text{seconds} \times 2.5 \; \text{GHz} = 10 \; \text{seconds} \times (2.5 \times 10^9 \; \text{cycles per second}) = 2.5 \times 10^{10} \; \text{cycles}\)

\(\frac{2.5 \times 10^{10} \; \text{instructions}}{1 \; \text{seconds}} = 2.5 \times 10^{10} \; \text{instructions per second}\)

\(10 \;\text{seconds} \times (2.5 \times 10^{10} \;\text{instructions per second}) = 2.5 \times 10^{11} \;\text{instructions}\)

\begin{center}\rule{0.5\linewidth}{\linethickness}\end{center}

\(P3 = 10 \; \text{seconds} \times 4 \; \text{GHz} = 10 \; \text{seconds} \times (4 \times 10^9 \; \text{cycles per second}) = 4 \times 10^{10} \; \text{cycles}\)

\(\frac{4 \times 10^{10} \; \text{instructions}}{2.2 \; \text{seconds}} = 1.82 \times 10^{10} \; \text{instructions per second}\)

\(10 \;\text{seconds} \times (1.82 \times 10^{10} \;\text{instructions per second}) = 1.82 \times 10^{11} \;\text{instructions}\)

\begin{center}\rule{0.5\linewidth}{\linethickness}\end{center}

\subsubsection{c.}\label{header-n4767}

\begin{quote}
We are trying to execute the same program as in part b), and we are
trying to reduce the execution time by 30\% but this leads to an
increase of 20\% in the CPI. What clock rate should we have to get this
time reduction?
\end{quote}

\begin{itemize}
\item
  Target: 70\% of old execution time (10 seconds), or 7 seconds
\end{itemize}

CPI of \textbf{P1} = 1.5, increase of 20\% = \(1.5 \times 1.2 = 1.8\)

\(\frac{2 \times 10^{11} \;\text{instructions}}{7 \;\text{seconds}} \times \frac{1.8 \;\text{cycles}}{1 \;\text{instruction}} = 0.51 \times 10^{11} \;\text{cycles per second} = 51 \; \text{GHz}\)

\begin{center}\rule{0.5\linewidth}{\linethickness}\end{center}

CPI of \textbf{P2} = 1, increase of 20\% = \(1 \times 1.2 = 1.2\)

\(\frac{2.5 \times 10^{11} \;\text{instructions}}{7 \;\text{seconds}} \times \frac{1.2 \;\text{cycles}}{1 \;\text{instruction}} = 0.43 \times 10^{11} \;\text{cycles per second} = 43 \; \text{GHz}\)

\begin{center}\rule{0.5\linewidth}{\linethickness}\end{center}

CPI of \textbf{P3} = 2.2, increase of 20\% = \(2.2 \times 1.2 = 2.64\)

\(\frac{1.82 \times 10^{11} \;\text{instructions}}{7 \;\text{seconds}} \times \frac{2.64 \;\text{cycles}}{1 \;\text{instruction}} = 0.69 \times 10^{11} \;\text{cycles per second} = 69 \; \text{GHz}\)

\begin{center}\rule{0.5\linewidth}{\linethickness}\end{center}

\subsection{2.}\label{header-n4790}

\begin{quote}
Consider two different implementations of the same instruction set
architecture. The instructions can be divided into four classes
according to their CPI (class A, B, C and D). P1 with a clock rate of
2.5 GHz and CPIs of 1, 2, 3 and 3, and P2 with a clock rate of 3 GHz and
CPIs of 2, 2, 2 and 2.
\end{quote}

\begin{quote}
Given a program with a dynamic instruction count of 1.0E6 instructions
divided into classes as follows: 10\% class A, 20\% class B, 50\% class
C and 20\% class D, which implementation is faster?
\end{quote}

\begin{longtable}[]{@{}lrrr@{}}
\toprule
Instruction class & P1 (2.5 GHz) CPI & P2 (3 GHz) CPI & Instruction
count\tabularnewline
\midrule
\endhead
A & 1 & 2 & 100,000\tabularnewline
B & 2 & 2 & 200,000\tabularnewline
C & 3 & 2 & 500,000\tabularnewline
D & 3 & 2 & 200,000\tabularnewline
\bottomrule
\end{longtable}

P1: \(2.5 \times 10^9\) cycles/sec

P1 (A): 1 cycle per instructions with 100,000 instructions = 100,000
cycles =\textgreater{}
\(100,000 \div \frac{2.5 \times 10^9 \;\text{cycles}}{1 \;\text{second}} = 0.00004\)
seconds

P1 (B): 2 cycles per instruction with 200,000 instructions = 400,000
cycles =\textgreater{}
\(400,000 \div \frac{2.5 \times 10^9 \;\text{cycles}}{1 second} = 0.00016\)
seconds

P1 (C): 3 cycles per instruction with 500,000 instructions = 1.5 million
cycles
=\textgreater{}\(1,500,000 \div \frac{2.5 \times 10^9 \;\text{cycles}}{1 \;\text{second}} = 0.0006\)
seconds

P1 (D): 3 cycles per instruction with 200,000 instructions = 600,000
cycles =\textgreater{}
\(600,000 \div \frac{2.5 \times 10^9 \;\text{cycles}}{1 \;\text{second}} = 0.00024\)
seconds

\textbf{P1 total} = \(0.00004 + 0.00016 + 0.0006 + 0.00024 = 0.00104\)
seconds

\begin{center}\rule{0.5\linewidth}{\linethickness}\end{center}

P2: \(3 \times 10^9\) cycles/sec

P2 (A): 2 cycles per instructions with 100,000 instructions = 200,000
cycles =\textgreater{}
\(200,000 \div \frac{2.5 \times 10^9 \;\text{cycles}}{1 \;\text{second}} = 0.00008\)
seconds

P2 (B): 2 cycles per instruction with 200,000 instructions = 400,000
cycles =\textgreater{}
\(400,000 \div \frac{2.5 \times 10^9 \;\text{cycles}}{1 \;\text{second}} = 0.00016\)
seconds

P2 (C): 2 cycles per instruction with 500,000 instructions = 1 million
cycles
=\textgreater{}\(1,000,000 \div \frac{2.5 \times 10^9 \;\text{cycles}}{1 \;\text{second}} = 0.0004\)
seconds

P2 (D): 2 cycles per instruction with 200,000 instructions = 400,000
cycles =\textgreater{}
\(600,000 \div \frac{2.5 \times 10^9 \;\text{cycles}}{1 \;\text{second}} = 0.00016\)
seconds

\textbf{P2 total} = \(0.00008 + 0.00016 + 0.0004 + 0.00016 = 0.0008\)
seconds

\textbf{P2 is a faster implementation}

\subsubsection{a.}\label{header-n4854}

\begin{quote}
What is the global CPI for each implementation?
\end{quote}

P1:
\((1 \times 0.1) + (2 \times 0.2) + (3 \times 0.5) +(3 \times 0.2) = 0.1 + 0.4 + 1.5 + 0.6 = 2.6\)

P2:
\((2 \times 0.1) + (2 \times 0.2) + (2 \times 0.5) +(2 \times 0.2) = 0.2 + 0.4 + 1 + 0.4 = 2\)

\subsubsection{b.}\label{header-n4862}

\begin{quote}
Find the clock cycles required in both cases.
\end{quote}

P1:
\(1,000,000 \;\text{instructions} \times \frac{2.6 \;\text{cycles}}{1 \;\text{instruction}} = 2,600,000 \;\text{clock cycles}\)

P2:
\(1,000,000 \;\text{instructions} \times \frac{2 \;\text{cycles}}{1 \;\text{instruction}} = 2,000,000 \;\text{clock cycles}\)

\subsection{3}\label{header-n4870}

\begin{quote}
Assume for arithmetic, load/store, and branch instructions, a processor
has CPIs of 1, 12, and 5, respectively. Also assume that on a single
processor, a program requires the execution of 2.56E9 arithmetic
instructions, 1.28E9 load/store instructions, and 256 million branch
instructions. Assume that each processor has a 2 GHz clock frequency.
\end{quote}

\begin{quote}
Assume that, as the program is parallelized to run over multiple cores,
the number of arithmetic and load/store instructions per processor is
divided by (0.7 * p) (where p is the number of processors) but the
number of branch instructions per processor remains the same.
\end{quote}

\begin{longtable}[]{@{}lrr@{}}
\toprule
Instruction Type & Instruction CPI & Instruction Count\tabularnewline
\midrule
\endhead
Arithmetic & 1 & 2,560,000,000\tabularnewline
Load/Store & 12 & 1,280,000,000\tabularnewline
Branch & 5 & 256,000,000\tabularnewline
\bottomrule
\end{longtable}

\subsubsection{a.}\label{header-n4898}

\begin{quote}
Find the total execution time for this program on 1, 2, 4 and 8
processors, and show the relative speedup of the 2, 4, and 8 processor
result relative to the single processor result.
\end{quote}

\paragraph{1 processor}\label{header-n4902}

\begin{itemize}
\item
  Arithmetic

  \begin{itemize}
  \item
    1 CPI * 2.6 billion instructions = 2.6 billion cycles
    =\textgreater{} 2.6 billion cycles / 2 GHz = 1.3 seconds
  \end{itemize}
\item
  Load/Store

  \begin{itemize}
  \item
    12 CPI * 1.28 billion instructions = 15.36 billion cycles
    =\textgreater{} 15.36 billion cycles / 2 GHz = 7.68 seconds
  \end{itemize}
\item
  Branch

  \begin{itemize}
  \item
    5 CPI * 256 million instructions = 1.28 billion cycles
    =\textgreater{} 1.28 billion cycles / 2 GHz = 0.64 seconds
  \end{itemize}
\item
  Total execution time: 1.3 seconds + 7.68 seconds + 0.64 seconds = 9.62
  seconds
\end{itemize}

\paragraph{2 processors}\label{header-n4928}

\begin{itemize}
\item
  Arithmetic

  \begin{itemize}
  \item
    1 CPI * (2.6 billion instructions / (0.7 * 2)) = 1,857,142,857
    cycles =\textgreater{} 0.928 seconds
  \item
    Speedup = 1.3 seconds / 0.928 seconds = 1.4X
  \end{itemize}
\item
  Load/Store

  \begin{itemize}
  \item
    12 CPI * (1.28 billion instructions / (0.7 * 2)) = 10,971,428,571
    cycles =\textgreater{} 5.485 seconds
  \item
    Speedup = 7.68 seconds / 5.485 seconds = 1.4X
  \end{itemize}
\item
  Branch

  \begin{itemize}
  \item
    No change (0.64 seconds)
  \item
    Speedup: 1X
  \end{itemize}
\item
  Total execution time: 0.928 seconds + 5.485 seconds + 0.64 seconds =
  7.053 seconds
\end{itemize}

\paragraph{4 processors}\label{header-n4963}

\begin{itemize}
\item
  Arithmetic

  \begin{itemize}
  \item
    1 CPI * (2.6B instructions / (0.7 * 4)) = 0.92B cycles
    =\textgreater{} 0.46 seconds
  \item
    Speedup = 1.3 seconds / 0.46 seconds = 2.83X
  \end{itemize}
\item
  Load/Store

  \begin{itemize}
  \item
    12 CPI * (1.28B instructions / (0.7 * 4)) = 5.485B cycles
    =\textgreater{} 2.74 seconds
  \item
    Speedup = 7.68 seconds / 2.74 seconds = 2.8X
  \end{itemize}
\item
  Branch

  \begin{itemize}
  \item
    No change (0.64 seconds)
  \item
    Speedup: 1X
  \end{itemize}
\item
  Total execution time: 0.46 seconds + 2.74 seconds + 0.64 seconds =
  3.84 seconds
\end{itemize}

\paragraph{8 processors}\label{header-n4998}

\begin{itemize}
\item
  Arithmetic

  \begin{itemize}
  \item
    1 CPI * (2.6B instructions / (0.7 * 8)) = 0.46B cycles
    =\textgreater{} 0.23 seconds
  \item
    Speedup = 1.3 seconds / 0.23 seconds = 5.65X
  \end{itemize}
\item
  Load/Store

  \begin{itemize}
  \item
    12 CPI * (1.28B instructions / (0.7 * 8)) = 2.74B cycles
    =\textgreater{} 1.37 seconds
  \item
    Speedup = 7.68 seconds / 1.37 seconds = 5.6X
  \end{itemize}
\item
  Branch

  \begin{itemize}
  \item
    No change (0.64 seconds)
  \item
    Speedup: 1X
  \end{itemize}
\item
  Total execution time: 0.23 seconds + 1.37 seconds + 0.64 seconds =
  2.24 seconds
\end{itemize}

\subsubsection{b.}\label{header-n5033}

\begin{quote}
If the CPI of the arithmetic instructions was doubled, what would the
impact be on the execution time of the program on 1, 2, 4 or 8
processors?
\end{quote}

The number of cycles and execution time for the arithmetic portion would
double.

\begin{itemize}
\item
  1 processor: Total execution time: (1.3 seconds * 2) + 7.68 seconds +
  0.64 seconds = 10.92 seconds
\item
  2 processors: Total execution time: (0.928 seconds * 2) + 5.485
  seconds + 0.64 seconds = 7.981 seconds
\item
  4 processors: Total execution time: (0.46 seconds * 2) + 2.74 seconds
  + 0.64 seconds = 4.3 seconds
\item
  8 processors: Total execution time: (0.23 seconds * 2) + 1.37 seconds
  + 0.64 seconds = 2.47 seconds
\end{itemize}

\subsubsection{c.}\label{header-n5052}

\begin{quote}
To what should the CPI of load/store instruction be reduced in order for
a single processor to match the performance of four processors using the
original CPI values.
\end{quote}

Total cycle count with 4 processors: 7.685B cycles

1 processor: (1 CPI * 2.6 billion instructions) + (x CPI * 1.28 billion
instructions) + (5 CPI * 256 million instructions)

7.685B cycles = 1 processor

7.685B cycles = (1 CPI * 2.6 billion instructions) + (x CPI * 1.28
billion instructions) + (5 CPI * 256 million instructions)

7.685B cycles = 2.6B cycles + (1.28B instructions)(x CPI) cycles + 1.28B
cycles

3.805B cycles = 1.28B * x

\textbf{x = 2.973 CPI}

Check: 2.973 CPI * 1.28B instructions = 3,805,440,000 cycles
=\textgreater{}3,805,440,000 cycles / 2 GHz = 1.90272 seconds

1.90272 seconds + 1.3 seconds + 0.64 seconds = 3.84 seconds = total
execution time for 4 processors

\subsection{4.}\label{header-n5074}

\begin{quote}
Consider a computer running a program that requires 250 s, with 70 s
spent executing float-point (FP) instructions, 85 s executed load/store
(L/S) instructions, and 40 s spent executing branch instructions and 55
s spent executing INT instructions.
\end{quote}

\subsubsection{a.}\label{header-n5078}

\begin{quote}
By how much is the total time reduced if the time for FP operations is
reduced by 20\%?
\end{quote}

70 s * .8 = 59.2 s =\textgreater{} total time - 10.797 s = 250 - 10.797
= 239.2 seconds

\subsubsection{b.}\label{header-n5084}

\begin{quote}
By how much is the time for INT operations reduced if the total time is
reduced by 20\%?
\end{quote}

Going off of initial percentages, FP = 28\%, L/S = 34\%, branch = 16\%,
INT = 22\%.

250 - 20\% = 200, 22\% of 200 = 44 seconds. 55 -44 = 11 second reduction

\subsubsection{c.}\label{header-n5092}

\begin{quote}
Can the total time be reduced by 20\% by reducing only the time for
branch instructions?
\end{quote}

250 -20\% = 200, delta = 50 seconds

Total branch time: 40s

40 seconds \textless{} 50 seconds so, therefore, no. This time reduction
can't be reached, even by eliminating branch instructions completely.

\subsection{5.}\label{header-n5102}

\begin{quote}
Assume that we are considering enhancing a machine by adding vector
hardware to it. When a computation is run in vector mode on the vector
hardware, it is 10 times faster than the normal mode of execution. We
call the percentage of time that could be spent using vector mode the
percentage of vectorization.
\end{quote}

\subsubsection{a.}\label{header-n5106}

\begin{quote}
Draw a graph that plots the speedup as a percentage of the computation
performed in vector mode. Label the y-axis ``Net Speedup'' and label
x-axis ``Percentage Vectorization.''
\end{quote}

\begin{figure}
\centering
\includegraphics{https://i.imgur.com/0jivKXZ.png}
\caption{}
\end{figure}

\subsubsection{b.}\label{header-n5112}

\begin{quote}
What percentage of vectorization is need to achieve a speedup of 2?
\end{quote}

S = speedup overall, F = vectorized fraction, P = vectorized speedup

\(S = \frac{1}{(1-F) +\frac{F}{P}}\)

\(S = 2\),

\(F = ?\),

\(P = 10\)

\(2 = \frac{1}{(1-F) + \frac{F}{10}}\)

\(2 = \frac{1}{1-0.9F}\)

\(2(1-0.9F) = 1\)

\(1 = 1.8F\)

\(F = 0.55 \;\; or \;\;55.55\%\)

\subsubsection{c.}\label{header-n5136}

\begin{quote}
What percentage of computation run time is spent in vector mode if a
speedup of 2 is achieved?
\end{quote}

Speedup of 2 means 55.55\% vectorization, 44.4\% normal (original total
time - F).

Speedup of 2 also means the original total time is halved.

halved total time - normal time = vectorized time.

Let original total time = 1000 seconds.

44.4\% of 1000 = 444 seconds

Halved time = 500 seconds.

500 - 444 = 56 seconds

56 / 1000 seconds = \textbf{5.6\% of time in vector mode}

\subsubsection{d.}\label{header-n5156}

\begin{quote}
What percentage of vectorization is needed to achieve one-half the
maximum speedup attainable from using vector model?
\end{quote}

Half the max speedup = 5.

\(5 = \frac{1}{1-0.9F}\)

\(5(1-0.9F) = 1\)

\(5 - 4.5F = 1\)

\(4 = 4.5F => F = 0.8888 = 88.89\%\)

\subsection{6.}\label{header-n5170}

\begin{quote}
Assume that we make an enhancement to a computer that improves some mode
of execution by a factor of 10. Enhanced mode is used 50\% of the time,
measure as a percentage of the execution time when the enhanced mode is
in use. Recall that Amdahl's law depends on the fraction of the
original, unenhanced execution time that could make use of enhanced
mode. Thus, we cannot directly use this 50\% measurement to compute
speedup with Amdahl's law.
\end{quote}

\subsubsection{a.}\label{header-n5174}

\begin{quote}
What is the speed up we have obtained from fast mode?
\end{quote}

50\% of time in normal mode, 50\% of time in enhanced mode.

Enhanced mode is 10 times faster than normal mode, so, to get the same
time fraction as normal mode, enhanced mode is actually 500\% in normal
mode, but sped up. Using this, the total time is 550\% in normal mode.

Using Amdahl's Law, speedup = exec. time of old / exec. time of new

\textbf{speedup = 550\% / 100\% = 5.5}

\subsubsection{b.}\label{header-n5186}

\begin{quote}
What percentage of the original execution time has been converted to
fast mode?
\end{quote}

S = speedup overall, F = vectorized fraction, P = vectorized speedup

\(S = \frac{1}{(1-F) +\frac{F}{P}}\)

\(S = 5.5\),

\(F = ?\),

\(P = 10\)

\(5.5 = \frac{1}{(1-F) + \frac{F}{10}}\)

\(5.5 = \frac{1}{1-0.9F}\)

\(5.5(1-0.9F) = 1\)

\(5.5 - 4.95F = 1\)

\(4.5 = 4.95F => F = 0.90909 = 90.9%\)

\end{document}
